\begin{titlepage}
	%\begin{minipage}{1\textwidth}
		\begin{center}
			\fontsize{12pt}{0.4\baselineskip}\selectfont \textbf{Министерство науки и высшего образования Российской Федерации \\ Федеральное государственное автономное образовательное учреждение \\ высшего образования \\ <<Московский государственный технический университет \\ имени Н.Э. Баумана \\ (национальный исследовательский университет)>> \\ (МГТУ им. Н.Э. Баумана)}
		\end{center}
	%\end{minipage}
%=======================================================
	\begin{center} % черная и серая полосы (это иллюзия -- они обе черные)
		\fontsize{12pt}{0.3\baselineskip}\selectfont
		\noindent\makebox[1.03\linewidth]{\rule{1.03\textwidth}{4pt}}
		         \makebox[1.03\linewidth]{\rule{1.03\textwidth}{4pt}}
	\end{center}
%=======================================================
	\begin{flushright}
		УТВЕРЖДАЮ\\
		Заведующий кафедрой ИУ7\\
		\raisebox{-0.7ex}{\makebox[1.1in]{\hrulefill}}\, Рудаков И.В.\\
		<<\makebox[0.3in]{\hrulefill}>> \makebox[0.9in]{\hrulefill}\, 20\makebox[0.3in]{\hrulefill}\, г.
	\end{flushright}
%=======================================================
\begin{center}
	\textbf{\fontsize{18pt}{0.8\baselineskip}\selectfont
	\textls[300]{ЗАДАНИЕ}\\
	\fontsize{16pt}{0.8\baselineskip}\selectfont
	на выполнение курсовой работы}\\
	по дисциплине\\
	\textbf{\fontsize{16pt}{\baselineskip}\selectfont
	Операционные системы}
\end{center}
%=======================================================
\fontsize{12pt}{0.8\baselineskip}\selectfont

\noindent Студент группы \textbf{ИУ7-75Б}
\vspace{-1.5ex}
\begin{center}
\textbf{Симонович Роман Дмитриевич}
\end{center}
\vspace{-1.5ex}

\noindent по теме
\vspace{-1.5ex}
\begin{center}
\textbf{Мониторинг выделения процессу физических страниц в Linux}
\end{center}
\vspace{-0.5ex}

\noindent Направленность КР (учебная, исследовательская, производственная, др.): учебная.

\noindent Источник тематики (кафедра, предприятие, НИР): кафедра.

\noindent График выполнения работы: 25\% к \uline{4} нед., 50\% к \uline{7} нед., 75\% к \uline{11} нед., 100\% к \uline{17} нед.

\noindent \textbf{Задание:}\\
\uline{Разработать загружаемый модуль ядра для синхронизации мониторинга страничных прерываний и количества выделенных процессу физических страниц.}

\noindent \textbf{Оформление курсовой работы:}\\
Расчетно-пояснительная записка на 30-40 листах формата А4.

\noindent Дата выдачи задания <<\raisebox{-0.7ex}{\makebox[0.25in]{\hrulefill}}>> \raisebox{-0.7ex}{\makebox[1in]{\hrulefill}} 2025г.
%=======================================================
	\begin{tabbing}
		Руководитель курсовой работы \hspace{2.9cm} \= \raisebox{-0.7ex}{\makebox[2in]{\hrulefill}}\, Рязанова Н. Ю.\\
        \>  \raisebox{0.5ex}{\makebox[2in][c]{\small\textit{подпись, дата}}} \\
		Студент \> \raisebox{-0.7ex}{\makebox[2in]{\hrulefill}}\, Симонович Р. Д.\\
		 \>  \raisebox{0.5ex}{\makebox[2in][c]{\small\textit{подпись, дата}}} \\
	\end{tabbing}
%=======================================================

\end{titlepage}
\addtocounter{page}{2}
