\subsection{Глава А}

В этой главе будут списки.

Ненумерованный список:
\begin{itemize}
	\item один;
	\item два;
	\item три.
\end{itemize}

Нумерованный список:
\begin{enumerate}
	\item один;
	\item два;
	\item три.
\end{enumerate}

Буквенный список:
\begin{asblist}
	\item один;
	\item два;
	\item три.
\end{asblist}

Текст текст текст.
Текст текст текст.
Текст текст текст.
Текст текст текст.
Текст текст текст.
Текст текст текст.

Далее идут формулы \ref{eq:f1} и \ref{eq:f2}:
\Eq{f1}{E = mc^2,}
\Eq{f2}{a^2 + b^2 = c^2.}

Следом пример таблицы \ref{tab:t1}.
\Table{
\begin{tabular}{|c|c|c|}
\hline
\textbf{Язык}   & \textbf{Синтаксис}                                               & \textbf{\begin{tabular}[c]{@{}c@{}}Встроенный\\ набор функций\end{tabular}} \\ \hline
\textbf{Java}   & 9/10                                                             & \multirow{2}{*}{Плюс минус одинаково}                                 \\ \cline{1-2}
\textbf{Python} & \begin{tabular}[c]{@{}c@{}}6/10\\ (чёрт ногу сломит)\end{tabular} &                                                                             \\ \hline
\end{tabular}
}{t1}{Сравнение языков программирования}

А в \ref{tab:t2} пример длинной таблицы.
\TableL{|c|c|c|c|c|}{t2}{Оценка каждого курса по конкретным параметрам}
\hline
\textbf{Курс} & \textbf{Предметы} & \textbf{Преподаватели} & \textbf{Рассписание} & \textbf{Экзамены} \\ \hline
\textbf{1}    & 8 / 10            & 7 / 10                 & 9 / 10               & 7 / 10            \\ \hline
\textbf{2}    & 7 / 10            & 5 / 10                 & 7 / 10               & 6 / 10            \\ \hline
\textbf{3}    & 5 / 10            & 4 / 10                 & 3 / 10               & 1 / 10            \\ \hline
\textbf{4}    & 7 / 10            & 5 / 10                 & 7 / 10               & 8 / 10            \\ \hline
\end{longtable}

Последняя колонка растягивается так, чтобы вся таблица была шириной в страницу. 
Это сделано для того, чтобы надпись "Продолжение таблицы" начиналась с левого края страницы (на самом деле, она начинается с левого края таблицы, в том то и проблема, но если левые границы страницы и таблицы совпадают, то и проблема не заметна).