% В основе преамбулы лежит: https://github.com/vpunch/gost732 (хотя от нее уже мало что осталось)
%===================================================================================
%===================================================================================

\documentclass[a4paper,14pt]{extarticle}

%===================================================================================
% Пакеты
\usepackage[utf8]{inputenc}            % кодировка исходного файла (UTF-8)
\usepackage[T2A]{fontenc}              % кодировка шрифтов для кириллицы
\usepackage[%
	left=30mm,
	right=15mm,
	top=20mm,
	bottom=20mm
]{geometry} 						   % поля страницы
\usepackage[english,russian]{babel}    % поддержка русского и английского языков
\usepackage{tempora}                   % шрифт Tempora (аналог Times)
\usepackage[onehalfspacing]{setspace}  % полуторный межстрочный интервал
\usepackage{indentfirst}               % отступ у первого абзаца после заголовков
\usepackage{etoc}                      % гибкая настройка оглавления
\usepackage{hyperref}                  % гиперссылки в документе
\usepackage{titlesec}                  % настройка оформления заголовков
\usepackage{graphicx}                  % вставка изображений
\usepackage{geometry}                  % управление геометрией страницы
\usepackage{totcount}                  % подсчёт элементов (страниц, рисунков и т.д.)
\usepackage{placeins}                  % управление размещением float-объектов
\usepackage{enumitem}                  % настройка списков
\usepackage[singlelinecheck=false]{caption} % настройка подписей к рисункам и таблицам
\usepackage{chngcntr}                  % управление нумерацией счётчиков
\usepackage{listings}                  % оформление исходного кода
\usepackage{xcolor}                    % расширенная работа с цветами
\usepackage{tabularx}                  % таблицы с автоматической шириной столбцов
\usepackage{multirow}                  % объединение строк в таблицах
\usepackage{array}                     % расширенные возможности таблиц
\usepackage{xparse}                    % удобное создание новых команд
\usepackage{longtable}                 % таблицы на несколько страниц
\usepackage{amsmath}                   % расширенные математические формулы
\usepackage{tikz, pgfplots}            % создание графиков и векторной графики
\usepackage{adjustbox}                 % масштабирование и выравнивание объектов
\usepackage{float}                     % расширенное управление float-объектами
\usepackage{url}                       % корректное отображение URL
\usepackage{csquotes}                  % умные кавычки и цитирование
\usepackage{lastpage}                  % ссылка на номер последней страницы
\usepackage{xassoccnt}                 % ассоциированные счётчики
\usepackage[%
    backend=biber,
    bibstyle=gost-numeric,             % стиль библиографии по ГОСТ
    sorting=none
]{biblatex}                            % работа с библиографией

%===================================================================================
% Собственные команды
\newcommand{\eline}{%	Конец строки
	\vspace{\baselineskip}
}

\newcommand{\emptypage}{%	Пустая страница
  	\newpage
	\mbox{}
	\newpage
}

\newcommand{\inspdf}[2]{%	% вставка pdg страниц
  \clearpage
  \thispagestyle{empty}
  \newgeometry{margin=0pt}
  \noindent\includegraphics[page=#2,width=0.99\textwidth]{#1}
  \restoregeometry
  \clearpage
}
%===================================================================================
% Настройка параметров оформления
\setcounter{secnumdepth}{5}		% глубина нумерации разделов
\setcounter{tocdepth}{3}		% глубина содержание
\setlength{\parindent}{1.25cm}	% отступ первой строки
\hypersetup{					% офрмление ссылок
    colorlinks=true, % false: ссылки в цветных прямоугольниках; true: цветные ссылки
    linkcolor=black,  % цвет внутренних ссылок
    filecolor=black, % цвет ссылок на файлы
    urlcolor=black,   % цвет URL-ссылок
    citecolor=black, % цвет ссылок на библиографию
    linkbordercolor={0 0 0} % цвет прямоугольников вокруг ссылок (чёрный)
}

\makeatletter
\renewcommand{\@pnumwidth}{15em}
\makeatother

%===================================================================================
% Оформление структурных элементов (глава, пунктов, приложений ...)
\titleformat{\section}
    [block]                           % форма
    {\filcenter\bfseries\normalsize}  % формат полностью
    {ПРИЛОЖЕНИЕ \thesection}          % метка
    {1em}                             % отступ от метки
    {}                                % код перед телом
% счётчик приложений
\newtotcounter{annexcount}
% приложение
\renewcommand{\thesection}{\Asbuk{section}}
\newcommand{\annex}[1]{%
    \stepcounter{annexcount}%
    \setcounter{figure}{0}
    \setcounter{table}{0}
    \setcounter{lstlisting}{0}
    \clearpage
    \section{#1}%
}

% приложение без названия
\newcommand{\annexIdx}{%
    \stepcounter{annexcount}%
    \setcounter{figure}{0}%
    \setcounter{table}{0}%
    \setcounter{lstlisting}{0}%
    \clearpage
    \section{\hspace{-0.1pt}}%
}
 
% скрытый структурный элемент
\newcommand{\hidedstructel}[1]{%
    \clearpage
    \section*{#1}%
}
% структурный элемент
\newcommand{\structel}[1]{%
    \hidedstructel{#1}
    \addcontentsline{toc}{section}{#1}%
}

% оформление раздела
\titleformat{\subsection}[hang]
{\normalfont\fontsize{18}{10}\bfseries\centering}{\thesubsection}{0.5em}{} % расстояние между № и названием
\titlespacing\subsection{\parindent}{\parskip}{1.5ex}  % 21 = 14 * 1.5 === полуторный интервал
\renewcommand{\thesubsection}{\arabic{subsection}}
% раздел
\newcommand{\sect}[1]{%
    \clearpage
    \setcounter{figure}{0}  % сбросить нумерацию внутри раздела
    \setcounter{table}{0}
    \setcounter{lstlisting}{0}
    \subsection{#1}
}

% оформление подраздела
\titleformat{\subsubsection}[hang]
{\bfseries\normalsize}{\thesubsubsection}{0.5em}{}
\titlespacing\subsubsection{\parindent}{1ex}{1ex}
% подраздел
%\usepackage{placeins}
\newcommand{\subsect}[1]{%
    \FloatBarrier
    \subsubsection{#1}
    \renewcommand{\theparagraph}{\thesubsubsection.\arabic{paragraph}}
}

% оформление пункта
\titleformat{\paragraph}[runin]
{\bfseries\normalsize}{\theparagraph}{0.5em}{} 
\titlespacing\paragraph{\parindent}{1ex}{1ex}
% пункт
\newcommand{\parag}{
    \paragraph{}
}

% оформление подпункта
\titleformat{\subparagraph}[runin]
{\bfseries\normalsize}{\thesubparagraph}{0.5em}{} 
\titlespacing\subparagraph{\parindent}{1ex}{1ex}
% пункт
\newcommand{\subparag}{
    \subparagraph{}
}

%===================================================================================
% Оформление списков
% общие параметры для списков
\setlist{
    topsep=0pt,					% отступ сверху и снизу списка
    partopsep=0pt,				% то же самое
    %leftmargin=0pt,			% отступ слева
    labelsep=0pt,				% отступ метки
    align=left,					% выравнивание метки
    listparindent=\parindent,	% отступ первой строки абзаца
    itemsep=0pt,				% отступ между элементами
    parsep=0pt					% отступ между абзацами и элементами
}
% параметры для нумерованных списков
\setlist[enumerate]{
    label={\arabic*)},
    labelwidth=1.4em%,
    %itemindent=\parindent+\labelwidth
    \setlength{\itemindent}{\labelwidth}
}
% параметры для ненумерованного списка
\setlist[itemize]{
    label=---~,  % в списках тире короткое, в тексте - длинное
    labelwidth=1.2em%,
    %itemindent=\parindent+\labelwidth
    \setlength{\itemindent}{\labelwidth}
}
% параметры для буквенного списка
\AddEnumerateCounter*{\asbuk}{\c@asbuk}{7}
\newlist{asblist}{enumerate}{2}
\setlist[asblist]{
    label={\asbuk*)},
    labelwidth=1.4em%,
    \setlength{\itemindent}{\labelwidth}
}

%===================================================================================
% подписи
\DeclareCaptionLabelSeparator{gost}{~---~}
\captionsetup{labelsep=gost}

% пакет для работы с рисунками
\graphicspath{{img/}}

% иллюстрация
\DeclareCaptionLabelFormat{gostfigure}{Рисунок #2}
\captionsetup[figure]{justification=centering, labelformat=gostfigure, position=bottom}
\newcommand{\fig}[3][1]{
	\renewcommand{\thefigure}{\thesubsection.\arabic{figure}}
    \begin{figure}[h]
        \centering
        \includegraphics[width=#1\textwidth]{#2}
        \caption{#3}\label{#2}
    \end{figure}
}

\newcommand{\comfig}[3][1]{
	\renewcommand{\thefigure}{\arabic{figure}}
    \begin{figure}[h]
        \centering
        \includegraphics[width=#1\textwidth]{#2}
        \caption{#3}\label{#2}
    \end{figure}
}

\newcommand{\annexfig}[3][1]{
	\counterwithin{figure}{section}
    \begin{figure}[h]
        \centering
        \includegraphics[width=#1\textwidth]{#2}
        \caption{#3}\label{#2}
    \end{figure}
}
%-----------------------------------------------------------------------------------
% листинг
% общие правила оформления кода
\lstset{ %
	language=SQL,					% выбор языка для подсветки	
	basicstyle=\small,				% размер и начертание шрифта для подсветки кода
	numbers=left,					% где поставить нумерацию строк (слева\справа)
	numberstyle=\small,				% размер шрифта для номеров строк
	stepnumber=1,					% размер шага между двумя номерами строк
	numbersep=5pt,					% как далеко отстоят номера строк от подсвечиваемого кода
	frame=single,					% рисовать рамку вокруг кода
	tabsize=4,						% размер табуляции по умолчанию равен 4 пробелам
	captionpos=t,					% позиция заголовка вверху [t] или внизу [b]
	breaklines=true,
	breakatwhitespace=true,			% переносить строки только если есть пробел
	escapeinside={\#*}{*)},			% если нужно добавить комментарии в коде
	xleftmargin=2em,				% Увеличение левого отступа
    %framexleftmargin=3.5em,		% Увеличение левого отступа для рамки
	backgroundcolor=\color{white}
}

% код в разделе
\DeclareCaptionLabelFormat{custlisting}{Листинг #2}
\captionsetup[lstlisting]{justification=raggedright, labelformat=custlisting, position=top}
\newcommand{\code}[2]{
	\renewcommand{\thelstlisting}{\thesubsection.\arabic{lstlisting}}
    \lstinputlisting[breaklines, caption=#2, label=#1]{#1}
}

% код вне раздела
\newcommand{\comcode}[2]{
	\renewcommand{\thelstlisting}{\arabic{lstlisting}}
    \lstinputlisting[breaklines, caption=#2, label=#1]{#1}
}

% код в приложении
\newcommand{\annexcode}[2]{
	\counterwithin{lstlisting}{section}
	\renewcommand{\thelstlisting}{\thesection.\arabic{lstlisting}}
    \lstinputlisting[breaklines, caption=#2, label=#1]{#1}
}

%-----------------------------------------------------------------------------------
% Оформление таблиц
% формат подписи к таблицы указывать самостоятельно перед begin(\tbl) в формате:
% \renewcommand{\thetable}{\arabic{table}}
\newenvironment{tbl}[3]
{
    \begin{table}[h]
        \small
        \centering
        \caption{#2}\label{tbl:#1}
        \begin{tabular}{|#3|}
            \hline
}{
            \hline
        \end{tabular}
    \end{table}
}
\DeclareCaptionLabelFormat{gosttable}{Таблица #2}
\captionsetup[table]{justification=raggedright, labelformat=gosttable, position=top}
\captionsetup[table]{labelsep=endash}

\newcommand{\specialcell}[2][c]{%
  \begin{tabular}[#1]{@{}c@{}}#2\end{tabular}}

% объединение строк
\newcommand{\mr}[2]{\multirow[t]{#1}{=}{#2}}

% колонки
\newcolumntype{M}[1]{>{\centering\arraybackslash}m{#1}}
\newcolumntype{N}[1]{>{\raggedright\arraybackslash}p{#1}}

% заголовок таблицы
%\usepackage{xparse}
\NewExpandableDocumentCommand\thead{t< t> O{1} m m}{%
    \IfBooleanTF{#1}{%
        \IfBooleanTF{#2}{%
            \multicolumn{#3}{|M{#4}|}{#5}%
        }{%
            \multicolumn{#3}{|M{#4}}{#5}%
        }
    }{%
        \IfBooleanTF{#2}{%
            \multicolumn{#3}{M{#4}|}{#5}%
        }{%
            \multicolumn{#3}{M{#4}}{#5}%
        }%
    }%
}

% длинная таблица
%\usepackage{longtable}
\newenvironment{longtbl}[3]
{
    \small
    \begin{longtable}{|#3|}
        \caption{#2}\label{tbl:#1}\\
        \hline
}{
        \hline
    \end{longtable}
}

%===================================================================================
% математика и графика

% математика
%\usepackage{amsmath}  % amsmath mathtools
\numberwithin{equation}{subsection}

% графики
%\usepackage{tikz, pgfplots}
\pgfplotsset{compat=newest}

%\usepackage{adjustbox}
%\usepackage{float}
%\usepackage{url}

%===================================================================================
% источники
%\usepackage{csquotes}
%\usepackage[%
%    backend=biber,
%    bibstyle=gost-numeric, % gost-numeric  utf8gost705u.bst ieee
%    sorting=none
%]{biblatex}
\addbibresource{bibliography.bib}
%\addbibresource{biblatex-examples.bib}
\newcommand{\showbib}{%
    \structel{СПИСОК ИСПОЛЬЗОВАННЫХ ИСТОЧНИКОВ}%
    \printbibliography[heading=none]%
}

% отступы в источниках
\defbibenvironment{bibliography}
    {\list
        {}
        {\setlength{\leftmargin}{0pt}%
         \setlength{\itemindent}{\parindent}%
         \setlength{\itemsep}{0pt}%
         \setlength{\parsep}{0pt}}}
    {\endlist}
    {\item
     \printtext[labelnumberwidth]{%
        \printfield{labelprefix}%
        \printfield{labelnumber}%
     }%
     \hspace{0.5em}}

% метка без точки
\DeclareFieldFormat{labelnumberwidth}{#1.}

%===================================================================================
% прочее

% номер последней страницы
%\usepackage{lastpage}

% счётчик источников
\newtotcounter{bibcount}
\AtEveryBibitem{
    \stepcounter{bibcount}%
}
% счётчики таблиц и рисунков
\newtotcounter{tblcount}
\DeclareAssociatedCounters{table}{tblcount}
\newtotcounter{figcount}
\DeclareAssociatedCounters{figure}{figcount}

%===================================================================================
%===================================================================================



\begin{document}

\hidedstructel{РЕФЕРАТ}
Данный документ можно считать руководством или документацией к преамбуле <<preamble\_nir>>.

\textit{Создано: 21.03.2023}


\begingroup

\parindent 0pt
\newlength{\pagewidth}\setlength{\pagewidth}{1.1em}

\newlength{\sectnum}\setlength{\sectnum}{8.4em}
\newlength{\ssectnum}\setlength{\ssectnum}{1em}
\newlength{\sssectnum}\setlength{\sssectnum}{2em}

\newlength{\sssectindent}\setlength{\sssectindent}{2em} 

\newcommand*{\entrybody}{%
    \raggedright%
    \etocname\nobreak%
    \leaders\etoctoclineleaders\hfill%
    \rlap{\makebox[\pagewidth][r]{\etocpage}}%
    \vspace{0.56em}% хак для отступа
}

\newlength\lenparboxsect
\setlength\lenparboxsect\textwidth
\addtolength\lenparboxsect{-\sectnum}
\addtolength\lenparboxsect{-\pagewidth}

\newlength\lensectparboxz
\setlength\lensectparboxz\textwidth
\addtolength\lensectparboxz{-\pagewidth}

\etocsetstyle{section}
    {}
    {\leavevmode\etocifnumbered{\leftskip \sectnum}{\leftskip 0pt}}
    {\normalsize\etocifnumbered%
        {\llap{\makebox[\sectnum][l]{ПРИЛОЖЕНИЕ \etocnumber}}%
            \parbox[t][][t]{\lenparboxsect}{\entrybody}}%
        {\parbox[t][][t]{\lensectparboxz}{\entrybody}}\par}
    {}

\newlength\lenparboxsubsect
\setlength\lenparboxsubsect\textwidth
\addtolength\lenparboxsubsect{-\ssectnum}
\addtolength\lenparboxsubsect{-\pagewidth}

\etocsetstyle{subsection}
    {}
    {\leavevmode\leftskip \ssectnum}
    {\normalsize\llap{\makebox[\ssectnum][l]{\etocnumber}}%
        \parbox[t][][t]{\lenparboxsubsect}{\entrybody}\par}
    {}

\newlength\lenleftskip
\setlength\lenleftskip\sssectnum
\addtolength\lenleftskip{\sssectindent}

\newlength\lenparbox
\setlength\lenparbox\textwidth
\addtolength\lenparbox{-\sssectnum}
\addtolength\lenparbox{-\pagewidth}
\addtolength\lenparbox{-\sssectindent}

\etocsetstyle{subsubsection}
    {}
    {\leavevmode\setlength{\leftskip}{\lenleftskip}\relax}
    {\normalsize\llap{\makebox[\sssectnum][l]{\etocnumber}}%
        \parbox[t][][t]{\lenparbox}{\entrybody}\par}
    {}

\etocsettocstyle{\hidedstructel{СОДЕРЖАНИЕ}}{}

\tableofcontents

\endgroup


\structel{ВВЕДЕНИЕ}
Настоящая преамбула несовершенна. Возможно, в ней ещё предстоит многое улучшить. Однако, для начала работы её достаточно.

Текст самого руководства может быть дополнен в будущем.

Любые вопросы решаются с использованием \textit{Яндекса}.

\begin{flushright}
Симонович Р.Д.\\
21.03.2023
\end{flushright}


\sect{Оформление содержания}
Для вставки содержания в текст документа, в нужное место в коде вставляется команда: \verb!\begingroup

\parindent 0pt
\newlength{\pagewidth}\setlength{\pagewidth}{1.1em}

\newlength{\sectnum}\setlength{\sectnum}{8.4em}
\newlength{\ssectnum}\setlength{\ssectnum}{1em}
\newlength{\sssectnum}\setlength{\sssectnum}{2em}

\newlength{\sssectindent}\setlength{\sssectindent}{2em} 

\newcommand*{\entrybody}{%
    \raggedright%
    \etocname\nobreak%
    \leaders\etoctoclineleaders\hfill%
    \rlap{\makebox[\pagewidth][r]{\etocpage}}%
    \vspace{0.56em}% хак для отступа
}

\newlength\lenparboxsect
\setlength\lenparboxsect\textwidth
\addtolength\lenparboxsect{-\sectnum}
\addtolength\lenparboxsect{-\pagewidth}

\newlength\lensectparboxz
\setlength\lensectparboxz\textwidth
\addtolength\lensectparboxz{-\pagewidth}

\etocsetstyle{section}
    {}
    {\leavevmode\etocifnumbered{\leftskip \sectnum}{\leftskip 0pt}}
    {\normalsize\etocifnumbered%
        {\llap{\makebox[\sectnum][l]{ПРИЛОЖЕНИЕ \etocnumber}}%
            \parbox[t][][t]{\lenparboxsect}{\entrybody}}%
        {\parbox[t][][t]{\lensectparboxz}{\entrybody}}\par}
    {}

\newlength\lenparboxsubsect
\setlength\lenparboxsubsect\textwidth
\addtolength\lenparboxsubsect{-\ssectnum}
\addtolength\lenparboxsubsect{-\pagewidth}

\etocsetstyle{subsection}
    {}
    {\leavevmode\leftskip \ssectnum}
    {\normalsize\llap{\makebox[\ssectnum][l]{\etocnumber}}%
        \parbox[t][][t]{\lenparboxsubsect}{\entrybody}\par}
    {}

\newlength\lenleftskip
\setlength\lenleftskip\sssectnum
\addtolength\lenleftskip{\sssectindent}

\newlength\lenparbox
\setlength\lenparbox\textwidth
\addtolength\lenparbox{-\sssectnum}
\addtolength\lenparbox{-\pagewidth}
\addtolength\lenparbox{-\sssectindent}

\etocsetstyle{subsubsection}
    {}
    {\leavevmode\setlength{\leftskip}{\lenleftskip}\relax}
    {\normalsize\llap{\makebox[\sssectnum][l]{\etocnumber}}%
        \parbox[t][][t]{\lenparbox}{\entrybody}\par}
    {}

\etocsettocstyle{\hidedstructel{СОДЕРЖАНИЕ}}{}

\tableofcontents

\endgroup!.

Разумеется, папка \textit{preamble\_nir} должна находится в рабочей директории.

Для корректного отображения содержания \LaTeX\; необходимо запустить \textbf{дважды}.

К сожалению, до сих пор не решена проблема оформления в содержании \textit{Приложений}. Поэтому, порядок работы следующий:
\begin{enumerate}
	\item Написать \verb!\annex{<Название приложения>}!;
	\item Запустить \LaTeX\; первый раз;
	\item Запустить \LaTeX\; второй раз;
	\item Написать \verb!\annex{!\textbf{\textbackslash\textbackslash}\verb!<Название приложения>}!;
\end{enumerate}

После этого \LaTeX\; \textbf{не запускать}. В противном случае, всплывет баг, и в содержании будет:
\begin{center}
Приложение А <\textit{новая строка}> <название приложения>
\end{center}


\sect{Работа с разделами}
Работа с разделами очень проста:
\begin{itemize}
	\item \verb!sect{<название раздела>}! -- создание раздела;
	\item \verb!subsect{<название подраздела>}! -- создание подраздела;
	\item Пункты, подпункты и списки вынесены в подразделы.
\end{itemize}

\subsect{Пункты и подпункты}
Дополнительно подраздел может быть разбит на пункты и подпункты.

\parag \textbf{Пункты}\\
Оформление пункта следующее:\\
\hspace*{10mm} \verb! \parag \textbf{<название пункта>}\\!

\parag \textbf{Подпункты}\\
Оформление подпункта следующее:\\
\hspace*{10mm} \verb! \subparag \textbf{<название подпункта>}\\!
\eline
\subparag Впрочем, название пункта/подпункта можно и не выделять жирным
\subparag Для отступа я использовал \verb!\hspace*{10mm}!\\
\subparag Пункты и подпункты в содержание не попадают

\eline
\subsect{Списки}

Нумерованный список: \verb!\begin{enumerate} \end{enumerate}!

Ненумерованный список: \verb!\begin{itemize} \end{itemize} !

Буквенный список: \verb!\begin{asblist} \end{asblist} !

Пункт любого из списков: \verb!\item!

Списки можно комбинировать.


\sect{Библиография}
Для создания библиографии нужна следующая последовательность команд:
\begin{enumerate}
	\item Запустить \textit{Быстрая сборка}
	\item Запустить \textit{biber}
	\item Запустить \textit{Быстрая сборка}
\end{enumerate}

Для ссылки на источник используется \verb!\cite[<lable источника>]!


\sect{Вставка объектов}
Речь в этом разделе идет о картинках, таблицах, листинге и ссылках на них.

\subsect{Картинки}
Команда: \verb!\fig[размер]{file}{подпись}!\\
lable -- file.

Картинки хранить в папке \textbf{img}.

\subsect{Листинги}
Команда: \verb!\lst{file.c}{подпись}!\\
lable -- file.


\subsect{Таблицы}
Команда:
\begin{verbatim}
\begin{tbl}{lable}
{подпись}
<Таблица. Её предется создавать самостоятельно. Примеры в Яндексе.>
\end{tbl}
\end{verbatim}

\subsect{Ссылки}
Для ссылки на любой из объектов синтаксис одинаковый \texttildelow \verb!\ref{lable}!
\textit{Примечание: для таблиц lable указывается как: tbl:lable}


\sect{Специальные символы}
Небольшая справка по спец. знакам.

Вставка файлов:\\
\hspace{10mm}\verb!\input{<название вставляемого документа (без расширения)>}!\\

Пробелы: \verb!\; \qquad \quad hspace{10mm}!


Формат текста: \verb!\textbf{} \textit, \underline() \verb||!

\sect{Математические формулы}
Информация по математике в \textit{Яндексе}. Здесь лишь небольшое количество примеров.

Складывая $a$ в квадрате с 
$b$ в квадрате, получаем 
$c$ в квадрате. Или 
излагая языком математики: 
$c^{2}=a^{2}+b^{2}$

\noindent
\TeX{} произносится как
$\tau\epsilon\chi$.\\[6pt]
100~м$^{3}$ воды.\\[6pt]
Это исходит от моего $\heartsuit$.

Плюс, можно использовать:\\
\verb!\begin{multline}\label{<lable>} <формула> \end{multline!

\structel{ЗАКЛЮЧЕНИЕ}
Удачи тебе, Роман из будущего.

\begin{flushright}
Симонович Р.Д. из прошлого\\
21.03.2023
\end{flushright}
\begin{figure}[h]

\centering

\includegraphics[width=0.25\linewidth]{./preamble_nir/signature.png}


\end{figure}

\end{document}