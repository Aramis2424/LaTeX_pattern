% В основе преамбулы лежит: https://github.com/vpunch/gost732 (хотя от нее уже мало что осталось)
%===================================================================================
%===================================================================================

\documentclass[a4paper,14pt]{extarticle}

%===================================================================================
% Пакеты
\usepackage[utf8]{inputenc}						% кодировка исходного файла (UTF-8)
\usepackage[T2A]{fontenc}						% кодировка шрифтов для кириллицы
\usepackage[%
	left=30mm,
	right=15mm,
	top=20mm,
	bottom=20mm
]{geometry}										% поля страницы
\usepackage[english,russian]{babel}				% поддержка русского и английского языков
\usepackage{tempora}							% шрифт Tempora (аналог Times)
\usepackage[onehalfspacing]{setspace}			% полуторный межстрочный интервал
\usepackage{indentfirst}						% отступ у первого абзаца после заголовков
\usepackage{etoc}								% гибкая настройка оглавления
\usepackage{hyperref}							% гиперссылки в документе
\usepackage{titlesec}							% настройка оформления заголовков
\usepackage{graphicx}							% вставка изображений
\usepackage{geometry}							% управление геометрией страницы
\usepackage{totcount}							% подсчёт элементов (страниц, рисунков и т.д.)
\usepackage{placeins}							% управление размещением float-объектов
\usepackage{enumitem}							% настройка списков
\usepackage[singlelinecheck=false]{caption}		% настройка подписей к рисункам и таблицам
\usepackage{listings}							% оформление исходного кода
\usepackage{chngcntr}							% управление нумерацией счётчиков
\usepackage{xcolor}								% расширенная работа с цветами
\usepackage{tabularx}							% таблицы с автоматической шириной столбцов
\usepackage{multirow}							% объединение строк в таблицах
\usepackage{array}								% расширенные возможности таблиц
\usepackage{xparse}								% удобное создание новых команд
\usepackage{longtable}							% таблицы на несколько страниц
\usepackage{amsmath}							% расширенные математические формулы
\usepackage{tikz, pgfplots}						% создание графиков и векторной графики
\usepackage{adjustbox}							% масштабирование и выравнивание объектов
\usepackage{float}								% расширенное управление float-объектами
\usepackage{url}								% корректное отображение URL
\usepackage{csquotes}							% умные кавычки и цитирование
\usepackage{lastpage}							% ссылка на номер последней страницы
\usepackage{xassoccnt}							% ассоциированные счётчики
\usepackage{booktabs}							% красивые горизонтальные линии
\usepackage[%
    backend=biber,
    bibstyle=gost-numeric,						% стиль библиографии по ГОСТ
    sorting=none
]{biblatex}										% работа с библиографией

%===================================================================================
% Собственные команды
\newcommand{\eline}{%	Конец строки
	\vspace{\baselineskip}
}

\newcommand{\emptypage}{%	Пустая страница
  	\newpage
	\mbox{}
	\newpage
}

\newcommand{\inspdf}[2]{%	% вставка pdg страниц
  \clearpage
  \thispagestyle{empty}
  \newgeometry{margin=0pt}
  \noindent\includegraphics[page=#2,width=0.99\textwidth]{#1}
  \restoregeometry
  \clearpage
}

%===================================================================================
% Настройка параметров оформления
\setcounter{secnumdepth}{5}		% глубина нумерации разделов

\setcounter{tocdepth}{4}		% глубина содержание ((1=section, 2=subsection, 3=subsubsection, 4=paragraph, 5=subparagraph)

\setlength{\parindent}{1.25cm}	% отступ первой строки

\hypersetup{					% офрмление ссылок
    colorlinks=true,		% false: ссылки в цветных прямоугольниках; true: цветные ссылки
    linkcolor=black,		% цвет внутренних ссылок
    filecolor=black,		% цвет ссылок на файлы
    urlcolor=black,			% цвет URL-ссылок
    citecolor=black,		% цвет ссылок на библиографию
    linkbordercolor={0 0 0}	% цвет прямоугольников вокруг ссылок (чёрный)
}

%===================================================================================
% Хуки и более тонкая настройка
\AddToHook{cmd/subsection/before}{\clearpage} % Раздел с новой страницы
\AddToHook{cmd/subsubsection/before}{\FloatBarrier} % Барьер для рисунков, таблиц перед подразделом

% Сброс счетчиков внутри раздела
\AtBeginDocument{%
  \counterwithin{figure}{subsection}
  \counterwithin{table}{subsection}
  \counterwithin{lstlisting}{subsection}
}

%-----------------------------------------------------------------------------------
% Оформление подписей к рисункам и таблицам с длинным тире
\DeclareCaptionLabelSeparator{gost}{~---~}
\captionsetup{labelsep=gost}

% Оформление подписей к рисункам
\DeclareCaptionLabelFormat{gostfigure}{Рисунок #2}
\captionsetup[figure]{justification=centering, labelformat=gostfigure, position=bottom}

% Оформление подписей к листингам
\DeclareCaptionLabelFormat{custlisting}{Листинг #2}
\captionsetup[lstlisting]{justification=raggedright, labelformat=custlisting, position=top}

% Оформление подписей к таблицам
\DeclareCaptionLabelFormat{gosttable}{Таблица #2}
\captionsetup[table]{justification=raggedright, labelformat=gosttable, position=top}

% Оформление нумерации списка источников
\DeclareFieldFormat{labelnumberwidth}{#1.}

%===================================================================================
% Оформление структурных элементов (глава, пунктов, приложений, ...)
%-----------------------------------------------------------------------------------
% Приложение
\titleformat{\section}
    [block]								% форма
    {\filcenter\bfseries\normalsize}	% формат полностью
    {ПРИЛОЖЕНИЕ \thesection}			% метка
    {0.5em}								% отступ от метки
    {}									% код перед телом

% Приложение с названием \appendX
\renewcommand{\thesection}{\Asbuk{section}}
\newcommand{\appendX}[1]{%
    \stepcounter{appendXcount}%
    \setcounter{figure}{0}
    \setcounter{table}{0}
    \setcounter{lstlisting}{0}
    \clearpage
    \section{#1}%
}

% Приложеним без названия \appendXn
\newcommand{\appendXn}{%
    \stepcounter{appendXcount}%
    \setcounter{figure}{0}%
    \setcounter{table}{0}%
    \setcounter{lstlisting}{0}%
    \clearpage
    \section{\hspace{-0.5pt}}%
}

%-----------------------------------------------------------------------------------
% Структурный элемент, не попадающий в содержание
\newcommand{\hidedstructel}[1]{%
    \clearpage
    \section*{#1}%
}

% Структурный элемент
\newcommand{\structel}[1]{%
    \hidedstructel{#1}
    \addcontentsline{toc}{section}{#1}%
}
%-----------------------------------------------------------------------------------
% Оформление раздела
\titleformat{\subsection}[hang]						% hang -- после название раздела с новой строки, runin -- на той же
  {\normalfont\fontsize{18}{21}\bfseries\centering}	% (21 = 14 * 1.5 === полуторный интервал)
  {\thesubsection}
  {0.5em}											% расстояние между № и названием
  {}				
\titlespacing\subsection{\parindent}{\parskip}{1.5ex}	% отступы заголовка
\renewcommand{\thesubsection}{\arabic{subsection}}		% Оформление нумерации разделов
% Раздел \subsection{}

%-----------------------------------------------------------------------------------
% Оформление подраздела
\titleformat{\subsubsection}[hang]
  {\bfseries\normalsize}	% стиль заголовка
  {\thesubsubsection}		% как отображается номер секции
  {0.5em}					% отступ между номером и названием
  {}						% код перед названием (например, "§ ")

% Подраздел \subsubsection{}

%-----------------------------------------------------------------------------------
% Оформление пункта
\titleformat{\paragraph}[hang]
  {\bfseries\normalsize}
  {\theparagraph}
  {0.5em}
  {}
\titlespacing\paragraph{\parindent}{1ex}{1ex} % Настройка отступов (слева, сверху, снизу)
% Пункт \paragraph{}

% Оформление подпункта
\titleformat{\subparagraph}[hang]
  {\bfseries\normalsize}
  {\thesubparagraph}
  {0.5em}
  {}
\titlespacing\subparagraph{\parindent}{1ex}{1ex}
% Подпункт \subparagraph{}

%===================================================================================
% Оформление списков
% Общие параметры для списков
\setlist{
    topsep=0pt,					% отступ сверху и снизу списка
    partopsep=0pt,				% то же самое
    %leftmargin=0pt,			% отступ слева
    labelsep=0pt,				% отступ метки
    align=left,					% выравнивание метки
    listparindent=\parindent,	% отступ первой строки абзаца
    itemsep=0pt,				% отступ между элементами
    parsep=0pt					% отступ между абзацами и элементами
}
% Параметры для нумерованных списков
\setlist[enumerate]{
    label={\arabic*)},
    labelwidth=1.4em
    \setlength{\itemindent}{\labelwidth}
}
% Параметры для ненумерованного списка
\setlist[itemize]{
    label=---~,  % Длинное тире
    labelwidth=1.2em
    \setlength{\itemindent}{\labelwidth}
}
% Параметры для буквенного списка
\AddEnumerateCounter*{\asbuk}{\c@asbuk}{7}
\newlist{asblist}{enumerate}{2}
\setlist[asblist]{
    label={\asbuk*)},
    \setlength{\itemindent}{\labelwidth}
}

%===================================================================================
% Оформление рисунков

\graphicspath{{img/}} % Директория по умолчанию

% Рисунок в разделе
\NewDocumentCommand{\Fig}{ O{1} m m }{% [размер] файл/label подпись
    \counterwithin{figure}{subsection}%
    \begin{figure}[!htbp]
        \centering
        \includegraphics[width=#1\textwidth]{#2}
        \caption{#3}
        \label{fig:#2}
    \end{figure}
}
% \Fig[0.8]{label01}{Подпись}

% Рисунок со сквозной нумерацией
\NewDocumentCommand{\GlobalFig}{ O{1} m m }{%
    \counterwithout{figure}{subsection}%
    \counterwithout{figure}{section}%
    \begin{figure}[!htbp]
        \centering
        \includegraphics[width=#1\textwidth]{#2}
        \caption{#3}
        \label{fig:#2}
    \end{figure}
}
%\GlobalFig{label02}{Подпись}

% Рисунок в приложении
\NewDocumentCommand{\AppendixFig}{ O{1} m m }{%
    \counterwithin{figure}{section}%
    \begin{figure}[!htbp]
        \centering
        \includegraphics[width=#1\textwidth]{#2}
        \caption{#3}
        \label{fig:#2}
    \end{figure}
}
% \AppendixFig{label03}{Подпись}

%===================================================================================
% Оформление листинга

% Общие правила оформления кода
\lstset{ %
	language=C,						% выбор языка для подсветки	
	basicstyle=\small,				% размер и начертание шрифта для подсветки кода
	numbers=left,					% где поставить нумерацию строк (слева\справа)
	numberstyle=\small,				% размер шрифта для номеров строк
	stepnumber=1,					% размер шага между двумя номерами строк
	numbersep=5pt,					% как далеко отстоят номера строк от подсвечиваемого кода
	frame=single,					% рисовать рамку вокруг кода
	tabsize=4,						% размер табуляции по умолчанию равен 4 пробелам
	captionpos=t,					% позиция заголовка вверху [t] или внизу [b]
	breaklines=true,
	breakatwhitespace=true,			% переносить строки только если есть пробел
	escapeinside={\#*}{*)},			% если нужно добавить комментарии в коде
	xleftmargin=2em,				% Увеличение левого отступа
    %framexleftmargin=3.5em,		% Увеличение левого отступа для рамки
	backgroundcolor=\color{white}
}

% Листинг в разделе
\NewDocumentCommand{\Code}{ O{C} m m m }{% [язык] файл label подпись
    \counterwithin{lstlisting}{subsection}%
    \lstinputlisting[
        language=#1,
        caption={#4},
        label={lst:#3}
    ]{#2}
}
% \Code{file.c}{label01}{Подпись}

% Листинг со сквозной нумерацией
\NewDocumentCommand{\GlobalCode}{ O{C} m m m }{%
    \counterwithout{lstlisting}{subsection}%
    \counterwithout{lstlisting}{section}%
    \lstinputlisting[
        language=#1,
        caption={#4},
        label={lst:#3}
    ]{#2}
}
% \GlobalCode[java]{file.java}{label01}{Подпись}

%Листинг в приложении
\NewDocumentCommand{\AppendixCode}{ O{C} m m m }{%
    \counterwithin{lstlisting}{section}%
    \lstinputlisting[
        language=#1,
        caption={#4},
        label={lst:#3}
    ]{#2}
}
% \AppendixCode{file.c}{label01}{Подпись}

%===================================================================================
% Оформление таблиц
% Для генерации таблиц использовать сайт https://tablesgenerator.com

% Таблица в разделе
\NewDocumentCommand{\Table}{ m m m }{% таблица label подпись
    \counterwithin{table}{subsection}%
    \begin{table}[!htbp]
        \centering
		\caption{#3}
        #1
        \label{tab:#2}
    \end{table}
}

% Таблица со сквозной нумерацией
\NewDocumentCommand{\GlobalTable}{ m m m }{%
    \counterwithout{table}{subsection}%
    \counterwithout{table}{section}%
    \begin{table}[!htbp]
        \centering
		\caption{#3}
        #1
        \label{tab:#2}
    \end{table}
}

% Таблица в приложении
\NewDocumentCommand{\AppendixTable}{ m m m }{%
    \counterwithin{table}{section}%
    \begin{table}[!htbp]
        \centering
		\caption{#3}
        #1
        \label{tab:#2}
    \end{table}
}

%===================================================================================
% Оформление математических формул
% Для генерации формул использовать сайт https://latexeditor.lagrida.com

% Формула в разделе
\NewDocumentCommand{\Eq}{ m m }{% label формула
    \counterwithin{equation}{subsection}%
	\label{eq:#1}
    \begin{equation}
    #2
    \end{equation}
}
% \Eq{f1}{E = mc^2}

% Формула со сквозной нумерацией
\NewDocumentCommand{\GlobalEq}{ m m }{%
    \counterwithout{equation}{subsection}%
    \counterwithout{equation}{section}%
    \label{eq:#1}
    \begin{equation}
    #2
    \end{equation}
}
% \GlobalEq{f1}{\int_0^\infty e^{-x^2} dx = \frac{\sqrt{\pi}}{2}}

% Формула в приложении
\NewDocumentCommand{\AppendixEq}{ m m }{%
    \counterwithin{equation}{section}%
    \label{eq:#1}
    \begin{equation}
    #2
    \end{equation}
}
% \AppendixEq{f1}{a^2 + b^2 = c^2}

%===================================================================================
% Оформление списка литературы

\addbibresource{bibliography.bib} % Файл со списком источников

\newcommand{\showbib}{%
    \structel{СПИСОК ИСПОЛЬЗОВАННЫХ ИСТОЧНИКОВ}%
    \printbibliography[heading=none]%
}

% отступы в источниках
\defbibenvironment{bibliography}
    {\list
        {}
        {\setlength{\leftmargin}{0pt}%
         \setlength{\itemindent}{\parindent}%
         \setlength{\itemsep}{0pt}%
         \setlength{\parsep}{0pt}}}
    {\endlist}
    {\item
     \printtext[labelnumberwidth]{%
        \printfield{labelprefix}%
        \printfield{labelnumber}%
     }%
     \hspace{0.5em}}

%===================================================================================
% Счётчики

% Счётчик источников
\newtotcounter{bibcount}
\AtEveryBibitem{
    \stepcounter{bibcount}%
}

% Счётчики таблиц
\newtotcounter{tblcount}
\DeclareAssociatedCounters{table}{tblcount}

% Счётчики рисунков
\newtotcounter{figcount}
\DeclareAssociatedCounters{figure}{figcount}

% Счётчик приложений
\newtotcounter{appendXcount} % счётчик приложений

%===================================================================================
%===================================================================================

% Шаблоны
%-----------------------------------------------------------------------------------
\iffalse
% Шаблон рисунка
\begin{figure}[!htbp]
    \centering
    \includegraphics[width=0.8\textwidth, angle=90]{img/filename}
    \caption{Название рисунка}
    \label{fig:example}
\end{figure}
%
\fi
%-----------------------------------------------------------------------------------
\iffalse
% Шаблон листинга
\lstinputlisting[
        language=C,
        caption={Подпись},
        label={lst:code01}
    ]{code/file.c}
%
\fi

\iffalse
% Шаблон листинга (без файла)
\begin{lstlisting}
def factorial(n):
    if n == 0:
        return 1
    else:
        return n * factorial(n-1)
print(factorial(5))
\end{lstlisting}
%
\fi
%-----------------------------------------------------------------------------------
\iffalse
% Шаблон таблицы
\Table{
\begin{tabular}{|l|l|ll|}
\hline
№ & \begin{tabular}[c]{@{}l@{}}Возможность\\ Многострочных фраз\end{tabular} & \multicolumn{1}{l|}{1}              & 2              \\ \hline
? & 4543                                                                     & \multicolumn{2}{l|}{Тогда жизнь определенно удалась} \\ \hline
\end{tabular}
}{t1}{Пример таблицы в разделе}
%
\fi
%-----------------------------------------------------------------------------------
